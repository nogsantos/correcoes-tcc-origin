\chaves{Intelig\^{e}ncia de neg\'{o}cios. Descoberta de conhecimento em banco de dados. Minera\c{c}\~{a}o de dados.}

\begin{resumo}

T\'{e}cnicas sobre a Intelig\^{e}ncia de Neg\'{o}cio t\^{e}m sido um importante t\'{o}pico de estudo. A descoberta de conhecimento em banco de dados e o processo de minera\c{c}\~{a}o de dados destaca-se por ser um conjunto de processos que une o uso do poder de processamento das m\'{a}quinas atuais, com avan\c{c}ados algoritmos de intelig\^{e}ncia artificial, projetados para analisar uma grande massa de dados com a finalidade de se descobrir informa\c{c}\~{o}es que at\'{e} ent\~{a}o eram desconhecidas ou desconsideradas. O objetivo deste estudo foi identificar as diferentes t\'{e}cnicas de utiliza\c{c}\~{a} e manipula\~{a}o de grandes massas de dados para a gera\c{c}\~{a}o de conhecimento com ganhos reais para indiv\'{i}duos e corpora\c{c}\~{o}es, com tempo e custo adequados. A metodologia empregada \'{e} um estudo bibliogr\'{a}fico, onde a busca por peri\'{o}dicos relevantes, livros e revistas cient\'{i}ficas sobre o tema foi realizada. O principal resultado deste estudo \'{e} identificar as ferramentas que permitem a descoberta do conhecimento, ap\'{o}s an\'{a}lise de grande volume de dados.

\end{resumo}

