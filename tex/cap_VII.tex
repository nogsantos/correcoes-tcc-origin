\chapter{Resultados}
\label{cap:resultados}

A minera��o de dados � uma das �reas do conhecimento mais deslumbrante no contexto da Tecnologia da Informa��o e Comunica��o. A principal contribui��o deste estudo foi a identifica��o das ferramentas que permitam a descoberta de conhecimento e a r�pida an�lise de um grande volume de informa��es, que cresce a cada dia nas organiza��es.

Utilizando ferramentas de \textit{software} desenvolvidas com esta finalidade e complexidade espec�fica, junto com conhecimento do funcionamento dos principais algoritmos de minera��o de dados, acredita-se que � poss�vel analisar e gerar conhecimento que at� ent�o n�o estava descoberto, junto � milhares ou milh�es de registros em bancos de dados empresarias, mostrando assim novas formas de atua��o a partir da descoberta feita.


