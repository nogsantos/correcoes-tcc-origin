\chapter{Conclus�o}
\label{cap:conclu}

A Minera\c{c}\~{a}o de dados � uma das �reas do conhecimento mais deslumbrante no contexto da Tecnologia da Informa��o e Comunica��o. A principal contribui��o deste estudo foi a identifica��o de ferramentas que permitam a descoberta de conhecimento e a r�pida an�lise em um grande volume de dados, que cresce a cada dia nas organiza��es ou at� mesmo na Internet.

Utilizando ferramentas desenvolvidas com esta finalidade e complexidade espec�fica, junto com conhecimento do funcionamento dos principais algoritmos de Minera\c{c}\~{a}o de dados, acredita-se que � poss�vel analisar e gerar conhecimento que at� ent�o estavam incobertos, junto � milhares ou milh�es de registros em bancos de dados corporativos, mostrando assim novas formas de atua��o a partir da descoberta feita.

Com a crescente massa de dados, torna-se evidente o uso de tecnologias mais avan�adas para o fornecimento de informa��es para apoiarem no processo de tomada de decis�o. Neste contexto, o uso de t�cnicas e algoritmos relacionados a Minera\c{c}\~{a}o de dados se fazem necess�rios para viabilizar a gera��o de conhecimentos e, dessa forma, fornecer subs�dios para os gestores maximizarem seus investimentos e alocarem da melhor forma seus recursos observando o custo-benef�cio. Para isto, percebe-se que existem um conjunto de t�cnicas, ferramentas e pr�ticas utilizadas em Intelig�ncia de neg�cio e Minera\c{c}\~{a}o de dados, cada qual com seus conceitos, boas pr�ticas, sistemas de computador e os algoritmos mais utilizados para estas finalidades. Assim, destaca-se a import�ncia das organiza��es tomarem conhecimento da sua realidade para que tenham sucesso e alcance os resultados almejados.